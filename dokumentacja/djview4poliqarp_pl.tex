\documentclass{mwart}
\usepackage{fontspec}
\usepackage{polyglossia}
\setmainlanguage{polish}

% https://unix.stackexchange.com/questions/162305/find-the-best-font-for-rendering-a-codepoint
% fc-search-codepoint
%\setmonofont[Mapping=tex-text]{FreeMono}
%\setmonofont[Mapping=tex-text]{DejaVuSansMono}
\setmonofont[Mapping=tex-text]{Junicode}

\usepackage{url}

\def\key#1{\textbf{#1}}

\title{djview4poliqarp}
\author{Janusz S. Bień}

\date{\today}

\begin{document}
\maketitle
\thispagestyle{empty}
\pagestyle{empty}

Program jest przede wszystkim graficznym zdalnym klientem serwera
oprogramowania korpusowego \textsf{Poliqarp for DjVu}; jako taki
stanowi modyfikację przeglądarki \textsf{djview4}
(\url{http://djvu.sourceforge.net/djview4.html}). Dodatkowo w 2017
r. program został rozbudowany o obsługę indeksów do dokumentów
w~formacie DjVu.

Postać kwerend służących do wyszukiwania w korpusach jest
udokumentowana w innych tekstach, między innymi
w~\url{http://nkjp.pl/poliqarp/help/pl.html} i
\url{http://poliqarp.sourceforge.net/man/poliqarp-query-syntax.html}.
Przykłady użycia można znaleźć m.in. w artykułach i prezentacjach
dostępnych do 30 czerwca 2018 r. w Bibliotece Cyfrowej Katedry
Lingwistyki Formalnej Uniwersytetu Warszawskiego
(\url{http://bc.klf.uw.edu.pl/}); ich lokalizacja po tej dacie nie
jest jeszcze znana.

\section{Konfiguracja}
\label{sec:konfiguracja}

\subsection{Serwery}
\label{sec:serwery}

Program jest dystrybuowany w postaci częściowo skonfigurowanej, w
szczególności dostępne są od razu adresy serwerów \textsf{Poliqarp for
  DjVu}, które oczywiście w razie potrzeby możemy zmienić w panelu
\texttt{Ustawienia --- Skonfiguruj --- Serwery}; adres serwera można
(trzeba?)  podać w postaci samej nazwy domenowej bez oznaczenia
protokołu.

Warto dodać, że oprogramowanie serwera i narzędzia do budowy korpusów
są dostępne na otwartych licencjach, zainteresowani mogą więc tworzyć
własne korpusy i serwery.

\subsection{Dodatkowe możliwości przeglądania plików DjVu}
\label{sec:dodatk-moliw-przegld}

Ponieważ funkcje programu są nieco inne niż przeglądarki
\textsf{djview4}, istnieje możliwość łatwego uruchomienia
\textsf{djview4} dla aktualnie oglądanego tekstu. Dla poprawnego
działania tej funkcji w wersji dla MS Windows jest niezbędne wskazanie
lokalizacji programu \textsf{djview4} w panelu \texttt{Ustawienia ---
  Skonfiguruj --- Ogólne}.

\subsection{Wygląd}
\label{sec:wygld}

Można wskazać język interfejsu na 3 sposoby:
\begin{itemize}
\item język systemu
\item polski
\item angielski
\end{itemize}
Efekt zmiany jest widoczny dopiero po ponownym uruchomieniu programu.

Można też wybrać kolor wyróżnienia wyników wyszukiwania.
% a co z innymi?

\subsection{Font}
\label{sec:font}

Można wybrać font używany do wyswietlania wyników tekstowych i
niektórych innych informacji w celu zmiany jego rozmiaru lub
repertuaru znaków. Rekomendowane są fonty zgodne z rekomendacjami
Medieval Unicode Font Initiative, np. \textsl{Junicode}.

Niektóre informacje są wyświetlane jednak zawsze z wykorzystaniem
domyślnego fontu systemu.

\subsection{Podgląd graficzny}
\label{sec:podgld-graficzny}

Można zmienić domyślną wysokość okienka podglądu wyników w panelu
bocznym.

Mozna również zmienić domyślną wartość powiększenia lupy.

\subsection{Dokument powitalny}
\label{sec:dokument-powitalny}

Można wskazać dokument, który zostanie wyświetlony w panelu głównym po
uruchomieniu programu. Możliwość w praktyce przydatna w razie
dystrybucji programu z odpowiednim dokumentem --- dotąd z tej
możliwości nie korzystano.

\subsection{Wprowadzanie znaków specjalnych}
\label{sec:wprow-znak-specj}
Można zdefiniować podstawienia ułatwiające wprowadzenia do pola
kwerend i do indeksu znaków niedostępnych na klawiaturze. Na przykład
poniższe podstawienie
\begin{verbatim}
/#=⸗
\end{verbatim}
umożliwia wprowadzanie znaku \texttt{DOUBLE OBLIQUE HYPHEN} jako
\texttt{/\#}. Analogicznie znak \texttt{SOFT HYPHEN} można wprowadzać
jako np. \texttt{/-}.

\subsection{Korzystanie z plików lokalnych}
\label{sec:korzystanie-z-plikow}

W celu przyspieszenia działania programu i uniezależnienia się od
serwera korpusu (jest to w pelni możliwe tylko przy korzystaniu z
samego indeksu) można pobrać odpowiednie dokumenty na dysk lokalny i
podać ich lokalizację za pomocą odpowiedniego podstawienia. Na
przykład jeśli na komputerze z systemem Linux pierwszy tom słownika
Lindego znajduje się w katalogu \texttt{/var/cache/Linde/1}, to
odpowiednie podstawienie powinno mieć postać
\begin{verbatim}
http://korpusy.klf.uw.edu.pl/djvus/linde-t/01/=file:///var/cache/Linde/1/
\end{verbatim}

Należy pamietać, że w przypadku słowników mamy z reguły do czynienia z
tzw. dokumentami rozłożonymi. Najprostszy sposób ich pobrania to
otworzenie w programie \textsf{djview} za pomocą \texttt{Open
  Location}, a następnie użycie \texttt{Save As}.

\subsection{Indeksy}
\label{sec:indeksy} 

Obsługa korpusów pojawiła się w 2017 r. w wersji 1.3. Początkowo
dokładnie jeden indeks był przypisany do konkretnego korpusu, obecnie
wybór indeksu lub indeksów jest niezależny od korpusu.

Indeksy to pliki tekstowe, hasła są reprezentowane przez wiersze
zawierające pola oddzielone średnikami. Początkowo indeksy miały trzy
pola, obecnie jest ich cztery (indeksy trzypolowe są nadal
akceptowane, czwarte pole traktuje się jako puste).

Aktualnie pola indeksu są następujące:
\begin{enumerate}
\item wyrażenie hasłowe --- służy do wyszukiwania i szeregowania
  haseł, wyświetlane w panelu listy haseł;
\item adres zaznaczenia w dokumencie DjVu, wyświetla się ono w panelu
  głównym po wybraniu hasła --- pole może być puste, wtedy wyrazenie
  hasłowe jest wyświetlone kursywą;
\item komentarz --- pole może być puste, po wybraniu hasla wyświetla
  sie pod panelem listy haseł (we wcześniej wersji hasło nie
  wyświetlało się, jesli pierwszym znakiem był wykrzyknik);
\item opis --- wyświetla się za wyrażeniem hasłowym, ale nie wpływa na
  ich porządkowanie czy wyszukiwanie, stosowane np. do róznicowania
  homonimów.
\end{enumerate}

Aktualnie dostępny jest indeks do słownika Lindego, patrz
\url{https://bitbucket.org/jsbien/ilindecsv}. Nowe indeksy można łatwo
tworzyć samodzielnie dzięki odpowiednim funkcjom programu. 

\section{Użytkowanie}
\label{sec:uytkowanie}

Sposób korzystania z programu jest dość oczywisty, najważniejsze
funkcje są dostępne w menu głównym oraz w podręcznych menu.

Skróty klawiaturowe są w dużym stopniu odziedziczone po programach
\textsf{djview4} i \textsf{djview3} Były one opisane w instrukcji do
tzw. wydania elektronicznego DjVuLibre słownika Knapskiego z 2005 r.;
ponieważ jest ona obecnie niedostępna, prawie bez zmian jest ona
powtórzona dalej (właściwy słownik jest nadal dostępny w e-BUW).
% http://ebuw.uw.edu.pl/publication/289334
% por. \url{http://old.mimuw.edu.pl/polszczyzna/Knapski/Knapski_DjVu/djvul.html}).
Zaproponowana wtedy terminologia nie jest --- jak się wydaje ---
powszechnie akceptowana.

\subsection{Panel główny}
\label{sec:panel-gowny}

Panel główny służy do przeglądania dokumentu DjVu, w szczególności do
oglądania wyniku wyszukiwania za pomocą kwerendy lub indeksu.

\begin{itemize}
  
\item Rozkład tekstu na ekranie:
  \begin{itemize}
  \item Ciągłe wyswietlanie stron: \key{F4}.
  \item Wyświetlanie stron obok siebie: \key{F5}.

    Wyświetlana para stron z reguły nie odpowiada tzw. rozkładówką
    oryginału: wyświetlają się strony 1-2, 3-4, a nie 2-3, 4-5 itd. W
    razie potrzeby wyswietlania rozkładówek należy skorzystać z
    programu \textsf{djview}, w którym taka funkcja jest dostępna.
  \end{itemize}
\item Skalowanie (ang. \textit{zoom}), większość poniższych operacji
  dostępna również z menu:
  \begin{itemize}
  \item Skalowanie względne
    \begin{itemize}
    \item na szerokość strony: \key{w} (od angielskiego \textit{Fit
\textbf{w}idth})
    \item na rozmiar strony: \key{p} (od angielskiego \textit{Fit
\textbf{p}age})
    \end{itemize}
  \item Skalowanie bezwzględne
    \begin{itemize}
    \item 100~\%: \key{1}
    \item 200~\%: \key{2}
    \item 300~\%: \key{3}
      % do sprawdzenia w źródłach!:
    \item kolejno 150~\%, 300~\%, 450~\% itp.: \key{+}
      % do sprawdzenia w źródłach!:
    \item kolejno 75~\%, 50~\%, 25~\% itp.: \key{-}
    \end{itemize}
  \item Skalowanie płynne: rolka myszy przy wciśnietym \key{Ctrl}.
  \item Lupa (ang. \textit{lens}): \key{Ctrl} razem z
    \key{Shift}. Lupa wyświetla zawsze obraz nawet jeśli włączone jest
    wyświetlanie tekstu ukrytego.
\end{itemize}
  \item Pozycjonowanie (ang. \textit{panning} i inne)
    \begin{itemize}
    \item Początek strony (górny lewy róg): \key{Home}
    \item Koniec strony (dolny lewy róg): \key{End}
    \item Przesunięcie strony w dół (kursora w górę): \key{Up}
    \item Przesunięcie strony w górę (kursora w dół): \key{Down}
    \item Przesunięcie strony w lewo (kursora w prawo): \key{Right}
    \item Przesunięcie strony w prawo (kursora w lewo): \key{Left}
    \item Przesunięcie strony myszą w dowolnym kierunku po naciśnięciu
lewego klawisza
   \item Skalowanie płynne w pionie: rolka myszy.
    \end{itemize}
  \item Przewijanie (ang. \textit{scrolling})
    \begin{itemize}
    \item Przewinięcie do przodu (tekst w górę, kursor w dół, w razie
potrzeby przejście do następnej strony): \key{Space}
    \item Przewinięcie do tyłu (tekst w dół, kursor w górę, w razie
potrzeby przejście do poprzedniej strony): \key{Backspace}
    \end{itemize}
  \item Nawigacja, poniższe operacje są dostępne również z menu:
    \begin{itemize}
    \item Następna strona (zachowuje aktualną pozycję strony w oknie): \key{Page Down}
    \item Poprzednia strona (zachowuje aktualną pozycję strony w oknie): \key{Page Up}
    \item Początek dokumentu: \key{Ctrl} + \key{Home}
    \item Koniec dokumentu: \key{Ctrl} + \key{End}
    \end{itemize}
  \item Obrót
    \begin{itemize}
    \item Obrót 90° w lewo (w kierunku przeciwnym do ruchu wskazówek
      zegara): \key{Ctrl}+\key{Left}.
    \item Obrót 90° w prawo (w kierunku zgodnym z ruchem wskazówek
      zegara): \key{Ctrl}+\key{Left}.
    \end{itemize}
  \item Zaznaczanie:
    \begin{itemize}
    \item Wciśnięcie klawisza \key{Ctrl} i lewego klawisza myszy
      pozwala zaznaczyć prostokątny fragment skanu, a następnie wybrać z
      menu podręcznego odpowiednią operację:
      \begin{itemize}
      \item Powiększ zaznaczenie
      \item ,,Kopiuj odnośnik'' czyli umieść w schowku adres tego zaznaczenia
      \item Kopiuj tekst ukryty pod zaznaczeniem
      \item Kopiuj zaznaczony fragment skanu
      \item Zapisz zaznaczony fragment skanu
      \item Dodaj hasło do indeksu (indeks musi być aktualnie otwarty,
        pierwsze pole jest wstępnie wypełnione przez tekst ukryty,
        drugie zawiera URL zaznaczenia)
      \item Uaktualnij hasło (URL zaznaczenia zastępuje odpowiednie pole w bieżącym haśle)
      \end{itemize}
    \end{itemize}
  % \item Usuwanie adnotacji:
  %   \begin{itemize}
  %   \item \key{h} (od ang. \textit{hide}); nie ma możliwości cofnięcia
  %     tej operacji.
  %   \end{itemize}
  \item Wyswietlanie tekstu ukrytego: \key{Ctrl}+\key{t}.
  \item Dostęp do fragmentu tekstu ukrytego:
    \begin{itemize}
    \item Wciśnięcie klawisza \texttt{Shift} wyświetla fragment tekstu
      ukrytego znajdujący się pod kursorem (fragment jest ograniczony
      przez strukturę tekstu zapisaną w dokumencie).
    \end{itemize}
  \end{itemize}
  Następujące operacje dostępne sa tylko z menu \textsf{Widok}:
  \begin{itemize}
  \item Wyswietlanie wierne (jednemu pikselowi dokumentu odpowiada
    jeden piksel na ekranie) --- \textsf{Wiernie}.
  \end{itemize}

\section{Panel kwerend}
\label{sec:panel-kwerend}

Panel ten wyświetla się z lewej strony ekranu; jego wyświetlanie można
włączyć i wyłączyć z menu \textsf{Widok}. Jego szerokość może być
zmieniana przez użytkownika za pomocą myszy.

Zawiera on następujące elementy:
\begin{itemize}
\item Pole serwera
\item Pole korpusu
\item Pole kwerend
\item Trzy podpanele:
  \begin{itemize}
  \item Podpanel wyników graficznych
  \item Podpanel wyników tekstowych
  \item Podpanel metadanych
  \end{itemize}
\item Przycisk \texttt{Więcej}
\item Informacja o liczbie trafień znalezionych i wyświetlonych.
\end{itemize}

Informacja o aktualnie znalezionych trafieniach może być
wyeksportowana w formacie cvs. Plik zawiera następujące pola
oddzielone przecinkami, w razie potrzeby ujęte w cudzysłowy:
\begin{enumerate}
\item numer trafienia
\item lewy kontekst
\item lewe uzgodnienie
\item prawe uzgodnienie
\item prawy kontekst
\item URL z zaznaczeniem trafienia
\item tekst kwerendy
\end{enumerate}

\subsection{Pole serwera}
\label{sec:pole-serwera}

Pole serwera pozwala wybrać konkretny serwer z listy ustalonej w
trakcie konfiguracji. Przycisk obok nazwy serwera wyswietla informację
o serwerze.

\subsection{Pole korpusu}
\label{sec:pole-korpusu}

Pole korpusu pozwala wybrać jeden z korpusów dostępnych na wybranym
wcześniej serwerze. Dodatkowo dostępne są dwa przyciski. Jeden z nich
wyświetla informację o korpusie, drugi pozwala zmienić domyślny sposób
wyszukiwania. Dostępne są nastepujące opcje (są one przekazywane do
programu \textsf{poliqarp}, w związku z tym są identyczne jak w innych
korpusach obsługiwanych przez ten program, np. Narodowy Korpus Języka
Polskiego):
\begin{itemize}
\item Ograniczenie przeszukiwania do losowej próbki i ustalenie jej wielkości.
\item Sortowanie wyników (domyślnie są wyswietlane w kolejności
  występowania w korpusie):
  \begin{itemize}
  \item według lewego lub prawego kontekstu, \textit{a fronte} lub
    \textit{a tergo};
  \item według ,,uzgodnienia'' (jeśli kwerenda jest podzielona na
    części, to według lewego uzgodnienia), \textit{a fronte} lub
    \textit{a tergo};
  \item według prawego ,,uzgodnienia'', jeśli kwerenda jest
    dwuczęściowa,
  \item według prawego kontekstu,
  \end{itemize}
\item Sposób wyświetlania elementów tekstu, określany niezależnie dla
  ,,uzgodnienia'' i kontekstu. Można włączyć lub wyłączyć
  wyświetlanie:
  \begin{itemize}
  \item formy pierwotnej (w typowych korpusach to forma fleksyjna),
  \item formy podstawowej (w typowych korpusach to forma hasłowa),
  \item znaczników (w typowych korpusach opisujacych własności gramatyczne)
  \end{itemize}
\item Długośc kontekstu w jednostkach wyświetlanego w konkordancjach i
  w oknie kontekstu.
\end{itemize}

\subsection{Pole kwerendy}
\label{sec:pole-kwerendy}

Program przekazuje kwerendy bez żadnych zmian do serwera, w związku z
czym ich opisu należy szukać w innych źródłach, które były wspomniane
we wstępie. Program przechowuje historię kwerend, zamiast wpisywać
kwerendę od początku można wybrać juz użytą kwerendę i w razie
potrzeby ją zmodyfikować..

Najprostsza kwerenda to po prostu napis, np. \texttt{broda}. Należy
pamiętać, że wielkość liter jest istotna, więc wyniki kwerend
\texttt{broda} i \texttt{Broda} są różne.

Dowolny element oznaczamy w kwerendzie jako \texttt{[]}. Oto jego
użycie w kwerendzie dwuczęściowej (wskazane jest wybranie sortowania
według prawego uzgodnienia): \verb|ku ^ []|.

\subsection{Podpanel wyników tekstowych}
\label{sec:podp-wynik-tekst}

Ten podpanel można uznać za główny, wyświetla wyniki w formie
konkordancji w sposób identyczny jak inne narzędzia wykorzystujące
serwer \textsf{poliqarp}.

Podwójne kliknięcie lewym klawiszem myszy na wiersz trafienia wyświetla jego obraz
graficzny w głównym panelu.

Pojedyncze kliknięcie kliknięcie lewym klawiszem myszy wybiera ten wiersz do następnych operacji:
\begin{itemize}
\item przejścia do okna metadanych, gdzie wyświetla się również dłuższy kontekst,
\item jednoczesne naciśnięcie klawisza \key{Ctrl} i \key{Backspace}
  usuwa z listy trafień ten wiersz jako nieinteresujący (nie ma
  możliwości cofnięcia tej operacjiw inny sposób niż ponowne wykonanie
  kwerendy).
\item przejścia do okna wyników graficznych z zachowaniem dokonanego
  wyboru (który może być niewidoczny).
\end{itemize}

Kliknięcie lewym klawiszem na numer trafienia wyświetla jego metadane
bez zmiany zawartości panelu głównego.

\subsection{Podpanel wynikow graficznych}
\label{sec:podp-wynik-graf}

Część funkcji jest analogiczna do panelu wyników tekstowych:

Podwójne kliknięcie lewym klawiszem myszy na wiersz trafienia wyświetla jego obraz
graficzny w głównym panelu.

Pojedyncze kliknięcie kliknięcie lewym klawiszem myszy wybiera ten wiersz do następnych operacji:
\begin{itemize}
\item przejścia do okna metadanych, gdzie wyświetla się również dłuższy kontekst,
\item jednoczesne naciśnięcie klawisza \key{Ctrl} i \key{Backspace}
  usuwa z listy trafień ten wiersz jako nieinteresujący (nie ma
  możliwości cofnięcia tej operacjiw inny sposób niż ponowne wykonanie
  kwerendy).
\item przejścia do okna wyników tekstowych z zachowaniem dokonanego
  wyboru (który może być niewidoczny).
\end{itemize}

Kliknięcie lewym klawiszem na numer trafienia wyświetla jego metadane
bez zmiany zawartości panelu głównego.

Kliknięcie środkowym klawiszem myszy na tekst w bocznym panelu
uruchamia nowy egzemplarz \textsf{djview4} wyświetlający odpowiednią
stronę; pozwala to wykonać na tekście operacje aktualnie nie dostępne
w \textsf{djview4poliqarp} --- np. wyświetlić tzw. konspekt
(\textit{outline}).

Każde z okienek podglądu graficznego działa podobnie, jak panel główny
--- możliwe jest w szczególności pozycjonowanie tekstu.

\subsection{Podpanel metadanych i kontekstu}
\label{sec:podpanel-metadanych-i}

Funkcja tego podpanelu jest oczywista, nie ma z nim związanych jakichś
specjalnych funkcji poza widocznymi przyciskami \texttt{Poprzedni},
\texttt{Następny}, \texttt{Usuń}. Jak w każdym panelu działa w nim
zaznaczanie i kopiowanie tekstu. Kliknięcie ha hiperlink uruchamia
\textsf{djview} z odpowiednim zaznaczeniem.

% \item Pojedyncze kliknięcie lewym klawiszem myszy na tekst w bocznym
%   panelu wybiera wskazany wynik i synchronizuje panele wyników
%   graficznych, tekstowych i metadanych.



\section{Panel indeksu}
\label{sec:panel-indeksu}

Panel ten wyświetla się z prawej strony ekranu; jego wyświetlanie
można włączyć i wyłączyć z menu \textsf{Widok}. Jego szerokość może
być zmieniana przez użytkownika za pomocą myszy.

Zawiera on następujące elementy:
\begin{itemize}
\item Informacja o nazwie indeksu, jego ewentualne modyfikacji i liczbie haseł.
\item Pole wyszukiwania przyrostowego haseł; domyślnie wyszukuje się
  od początku wyrażenie hasłowego, ale jeśli wprowadzony zapis zaczyna
  sie od znaku \texttt{*}, to wyszukuje się również podnapisy.
\item Lista haseł; normalnie obok hasła wyświetla się rownież jego
  opis (z wyjątkiem sortowania \textit{a tergo}).
\item Pole komentarza, które można edytować; po edycji można przejść
  do pola wyszukwiania za pomocą \key{Ctrl}+\key{f}.
\item Pole stanu: ścieżka do aktualnie otwartego indeksu, numer strony
  dokumentu wyświetlanej w panelu głównym,
\end{itemize}

Za pomocą menu głównego można wykonać następujące informacje na
indeksie:
\begin{itemize}
\item Otwórz wskazany plik jako indeks,
\item Otwórz ponownie jeden z ostatnio używanych indeksów
\item Odśwież aktualnie otwarty indeks (w razie dokonania zmian w
  pliku za pomocą innego programu)
\item Otwórz plik i dodaj jego zawartość do indeksu
\item Zapisz aktualnie otwarty indeks
\item Zamknij aktualnie otwarty indeks
\end{itemize} 

Po wpisaniu w polu wyszukiwania pewnego napisu naciśnięcie klawisza
\texttt{Enter} wyświetla cyklicznie kolejne hasła zaczynające sie od
tego napisu. Po liście haseł można poruszać się również za pomocą
strzałek w góre i w dół oraz klawiszy \texttt{Home} i \textup{End}.

Dostępne w menu i za pomocą skrótów klawiaturowych \key{Alt} +
strzałka w lewo i \key{Alt} + strzałka w prawo operacje przejścia do
poprzedniego i następnego hasła nie odnoszą się do kolejności na
liście haseł, ale do historii oglądania haseł.

W trakcie zapisywania zmodyfikowanego indeksu początkowa wersja
indeksu jest zachowana w formie kopii zapasowej. Jeśli otwartych było
kilka plików, zapisywane są one jako jeden plik.

Dodatkowe operacje są dostępne w menu podręcznym otwieranym przez
kliknięcie prawym klawiszem na wybrane hasło. Sa to dwie operacje
odnoszące się do wybranego hasła
\begin{itemize}
\item Edytuj hasło
\item Oznacz jako usunięte (hasło zostanie usunięte dopiero przy
  zapisywaniu indeksu)
\end{itemize}
oraz --- ze względów technicznych --- opcję odnoszącą sie do całego
indeksu, czyli ,,Kolejność haseł''. Umozliwia ona wybór następujących
sposobów ustalania kolejności:
\begin{itemize}
\item ,,Z pliku''
\item  Alfabetyczne \textit{a fronte} ,,słowo po słowie''
\item Alfabetyczne \textit{a fronte} ,,litera po literze''
\item \textit{A tergo}
\end{itemize}


\section{Uwagi końcowe}
\label{sec:uwagi-kocowe}

Tekst ten jest udostępniony na licencji GNU FDL i nie ma charakteru
ostatecznego. Żródła tekstu są dostępne w repozytorium
\url{https://bitbucket.org/jsbien/ndt}, uwagi i propozycje zmian można
zgłaszać za pomocą narzędzi dostępnych w tym repozytorium.

\end{document}
Do sprawdzenia:

  % \item Obsługa odnośników (ang. \textit{links})
  %   \begin{itemize}
  %   \item Użycie odnośnika: pstryknięcie lewym klawiszem myszy
  %   \item Pokazanie wszystkich odnośników na stronie: \key{Shift}
  %   \end{itemize}

%%% Local Variables: 
%%% coding: utf-8-unix
%%% mode: latex
%%% TeX-master: t
%%% TeX-PDF-mode: t
%%% TeX-engine: xetex
%%% End: 


